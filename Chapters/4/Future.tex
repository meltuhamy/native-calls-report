\section{Future Extensions} % (fold)
\label{sec:future_extensions}
This section discusses possible future extensions to the implementation of the project.


\subsection{Transferring contiguous number types as binary} % (fold)
\label{sub:transferring_binary}
In section \ref{sec:webidl_bindings} we discussed how to transfer a sequence of any type. This is represented in WebIDL as \lstinline{sequence<T>}, where \lstinline{T} could be any type, including dictionary types. But there is one case where it makes sense to send the data as binary data, through the use of ArrayBuffers. This is when we want to send a contiguous array of numeric type, for example, an array of floats. 

Sending binary data in that case is efficient for two reasons. The first is the fact you don't need to marshal the data into a \lstinline{pp::VarArray} type, since the binary buffer can be sent directly using the \lstinline{pp::VarArrayBuffer} class. The second reason is how binary data is transferred in NaCl. When we send an ArrayBuffer to/from JavaScript, instead of the data being copied, it is shared. Only when the data is written to does the data get copied. This makes transferring ArrayBuffers very efficient - instead of O(n) time, it will probably take O(1) time.

Now, considering the performance gains, if we decide to send and receive contiguous number arrays as ArrayBuffers, a few questions arise. The first is how will the data be represented in JavaScript, and whether or not this representation makes sense in every context. The answer is that in JavaScript, the data will need to be sent and received as an \lstinline{ArrayBuffer}. It's difficult to do anything with an ArrayBuffer though, so in JavaScript, a few more classes were made to help with reading buffers of certain types. These are called ArrayBufferViews. Currently available ArrayBufferViews are \lstinline{Int8Array}, \lstinline{Int16Array}, \lstinline{Int32Array}, \lstinline{Float32Array}, and \lstinline{Float64Array}, and also their unsigned counterparts. These classes allow accessing the data of a buffer as though it was a normal JavaScript array. So, when we relate these ArrayBufferViews to IDL types, these make sense for byte, short, long, float, and double WebIDL types. The \lstinline{long long} type will be unsupported, but that is understandable, considering JavaScript's number size limitations (as described earlier). In conclusion, the answer to the first question is ``the binary data will be represented in JavaScript as an appropriate ArrayBufferView, and this representation makes sense for most number array types''.

The second question is \emph{when} do we send binary data? To answer the question, we consider when it's possible to send arrays of numbers in general:

\begin{itemize}
	\item As a parameter
	\item As a result
	\item Embedded inside a dictionary or array
\end{itemize}

We could choose to send and receive binary for \emph{all} the above scenarios, or some. To figure out when to send, we need to run some benchmarks to find how much of a performance improvement it might give.

The third question is how do we accept binary data in C++? The possibilities are either to accept it as a buffer, or a vector. As discussed earlier, however, accepting it as a buffer is problematic since we need to provide the length of the array. Luckily, we can easily and efficiently construct a vector with the same data, by providing a pointer to the data in the constructor of the vector. When sending it back, we use the \lstinline{std::vector::data()} method to efficiently get a pointer to the buffer, that we can then use to send.

The fourth and final question we need to ask is how the data is transferred from C++ to JavaScript. The answer is through the \lstinline{pp::VarArrayBuffer} interface. But there arises a problem, to do with copying memory. To illustrate the problem, consider how we plan to send the array buffer:

\begin{itemize}
	\item In the server stub, the concrete function implementation is called, and the result - a \lstinline{std::vector} of numbers - is returned.
	\item A pointer to the buffer of the returned data is retrieved \lstinline{std::vector::data()}.
	\item The number of elements in the buffer is retrieved using \lstinline{std::vector::size()}.
	\item We now want to send the data pointed to by the buffer. To send the data, we need to use the \lstinline{pp::VarArrayBuffer} class. But this will just allocate its own memory.
	\item Problem: we already have the \lstinline{std::vector} allocating memory to the buffer we wish to send, but \lstinline{pp::VarArrayBuffer} can only be constructed by allocating its own memory.
\end{itemize}

One solution would be to copy the memory contents of the vector's buffer to the \lstinline{pp::VarArrayBuffer} buffer. But this seems like a slow solution.

Another solution is to specify the array length in the IDL file, and pass the result vector by reference to the function. So, before the function is called, a \lstinline{pp::VarArrayBuffer} is constructed with the correct number of bytes according to the length of the array. A pointer to that buffer is retrieved using \lstinline{pp::VarArrayBuffer::map()}. The pointer is used to construct a \lstinline{std::vector}. The \lstinline{std::vector} is passed by reference to the concrete function implementation. Instead of the function returning, it returns void but alters the \lstinline{std::vector}. When the function returns, the server stub unmaps the \lstinline{pp::VarArrayBuffer} again and sends it to JavaScript.
% subsection transferring_binary (end)
% section future_extensions (end)