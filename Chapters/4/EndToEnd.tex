\section{Getting Started Guide} % (fold)
\label{sec:end_to_end_example}
Now that we've discussed the design and implementation of the RPC Framework and generators, we show here a concrete example that illustrates how the implementation works, by referring to the relevant design section at each stage. This is in the form of the Native Calls getting started guide. The guide was published on GitHub in order to get anyone to easily use the Native Calls framework from scratch.

This guide shows how to create a simple C++ library using Native Calls.
We will create a complex number calculator using a C++ native module.
I've written this tutorial in a way such that you can follow along and
write the module yourself.

\hyperdef{}{writing-our-interface-using-web-idl}{\subsection{Writing our
interface using Web IDL}\label{writing-our-interface-using-web-idl}}

Native Calls works by generating JS and C++ that handles communication
between your native module and any JavaScript application. To do this,
you'll need to tell Native Calls what functions you want to expose to
JavaScript. You do this by writing the interface using Web IDL (which is
very simple). Native Calls then takes this IDL file and generates the
code for you.

Let's begin by creating a folder for our complex number calculator
project.

\begin{Shaded}
\begin{Highlighting}[]
\KeywordTok{mkdir} \NormalTok{complexCalculator}
\KeywordTok{cd} \NormalTok{complexCalculator}
\KeywordTok{vim} \NormalTok{complex.idl}
\end{Highlighting}
\end{Shaded}

Now, we write our complex number calculator IDL file
(\texttt{complex.idl}):

\begin{verbatim}
dictionary complex {
  double r;
  double i;
};

interface Calculator {
  complex add(complex x, complex y);
  complex subtract(complex x, complex y);
  complex multiply(complex x, complex y);
  complex sum_all(sequence<complex> contents);
  complex multiply_all(sequence<complex> contents);
  sequence<double> map_abs(sequence<complex> contents);
};
\end{verbatim}

Before moving on, let's take a closer look at the interface.

\subsubsection{Defining dictionary
types}\label{defining-dictionary-types}

Dictionary types get converted into C++ \texttt{struct}s. On JavaScript,
they define JavaScript objects. In the above, the \texttt{complex}
dictionary defines a struct in C++ that has the fields \texttt{r} and
\texttt{i}. It also defines the JavaScript object with the properties
\texttt{r} and \texttt{i}.

Once a dictionary is defined, it can be used as a type. Native Calls
allows many types, as defined in the
\href{http://www.w3.org/TR/WebIDL/}{Web IDL specification}.

\subsubsection{Defining interfaces}\label{defining-interfaces}

An interface can include definitions for many functions. These functions
will be exposed to the JavaScript (i.e.~we'll be able to call these
functions directly from Javascript). In the IDL above, we defined
\texttt{add}, \texttt{subtract} and \texttt{multiply} which all take two
paramaters of \texttt{complex} type and return a \texttt{complex} type.

Meanwhile, \texttt{sum\_all} takes a \texttt{sequence} type. Sequences
get converted into C++ \texttt{std::vector}s, and on JavaScript, they're
arrays.

\hyperdef{}{generating-the-rpc-module}{\subsection{Generating the RPC
module}\label{generating-the-rpc-module}}

Now that we've defined the interface for the module, we now pass it to
the generator. The generator lives in
\texttt{native-calls/generator/pprpcgen.js} and can be executed
directly. In the \texttt{complexCalculator} folder, generate the code
like this:

\begin{Shaded}
\begin{Highlighting}[]
\KeywordTok{~/native-calls/generator/pprpcgen.js} \NormalTok{--package=Complex complex.idl}
\end{Highlighting}
\end{Shaded}

\textbf{NOTE}: If you installed the package using \texttt{make install},
you should have the \texttt{pprpcgen} command installed globally. If so,
you can just type \texttt{pprpcgen -{}-package=Complex complex.idl}, and
use \texttt{pprpcgen} from any directory. You can also install only the
generator globally, without cloning the repo, by typing
\texttt{npm install -g native-calls}.

\texttt{pprpcgen} will create a folder called Complex (matching the
\texttt{-{}-package} option). Let's take a look inside.

\begin{Shaded}
\begin{Highlighting}[]
\KeywordTok{cd} \NormalTok{Complex}
\KeywordTok{ls}
\end{Highlighting}
\end{Shaded}

Using the IDL file, we can see that the generator generated the
following files:

\begin{itemize}
\itemsep1pt\parskip0pt\parsep0pt
\item
  \texttt{PPRPCGEN\_Calculator.h} This is the C++ interface that we need
  to implement
\item
  \texttt{ComplexRPC.cpp} This is the C++ RPC library, specific to our
  Complex number calculator
\item
  \texttt{ComplexRPC.js} The javascript file that we can include in our
  HTML to interface with the C++ library.
\item
  \texttt{PPRPCGEN\_ComplexTypes.h} Since we defined some extra types,
  (the \texttt{complex} dictionary type), this file is generated and
  includes the corresponding C++ \texttt{struct}.
\item
  \texttt{Makefile} Finally, a makefile is generated for us to be used
  as a template.
\end{itemize}

Take a look at each file to see how the RPC library works. Most
importantly let's see what's inside \texttt{Calculator.h} and
\texttt{ComplexTypes.h}.

\begin{Shaded}
\begin{Highlighting}[]
\KeywordTok{less} \NormalTok{PPRPCGEN_ComplexTypes.h}
\end{Highlighting}
\end{Shaded}

\begin{Shaded}
\begin{Highlighting}[]
\CommentTok{//...}
\KeywordTok{typedef} \KeywordTok{struct} \NormalTok{\{}
  \DataTypeTok{double} \NormalTok{r;}
  \DataTypeTok{double} \NormalTok{i;}
\NormalTok{\} complex;}
\end{Highlighting}
\end{Shaded}

As we expected, the dictionary was converted into an equivalent
\texttt{struct}.

\begin{Shaded}
\begin{Highlighting}[]
\KeywordTok{less} \NormalTok{PPRPCGEN_Calculator.h}
\end{Highlighting}
\end{Shaded}

\begin{Shaded}
\begin{Highlighting}[]
\OtherTok{#include "ComplexTypes.h"}
\OtherTok{#include "nativecalls/RPCType.h"}
\OtherTok{#include <vector>}
\KeywordTok{namespace} \NormalTok{pprpcgen\{}
\KeywordTok{namespace} \NormalTok{Calculator\{}
\NormalTok{complex add( complex x,  complex y);}

\NormalTok{complex subtract( complex x,  complex y);}

\NormalTok{complex multiply( complex x,  complex y);}

\NormalTok{complex sum_all( std::vector<complex> contents);}

\NormalTok{complex multiply_all( std::vector<complex> contents);}

\NormalTok{std::vector<}\DataTypeTok{double}\NormalTok{> map_abs( std::vector<complex> contents);}

\NormalTok{\}}
\NormalTok{\}}
\end{Highlighting}
\end{Shaded}

We can see that the generator created a header file for us to implement.
The header file is entirely standard C++, using no extra libraries.

\hyperdef{}{implementing-the-interface}{\subsection{Implementing the
interface}\label{implementing-the-interface}}

We can now start writing our implementation. The generated
\texttt{Makefile} requires us to write the implementation in a file
called \texttt{Calculator.cpp}, matching our interface name
(\texttt{Calculator.h}), in the same folder
(\texttt{\textasciitilde{}complexCalculator/Complex/}). Feel free to
skip over the actual implementation. I've placed it here so you can copy
it if you've been following along with the tutorial.

\begin{Shaded}
\begin{Highlighting}[]
\KeywordTok{vim} \NormalTok{Calculator.cpp}
\end{Highlighting}
\end{Shaded}

\begin{Shaded}
\begin{Highlighting}[]
\OtherTok{#include "PPRPCGEN_Calculator.h"}

\OtherTok{#include <complex>}
\OtherTok{#include <vector>}
\KeywordTok{namespace} \NormalTok{pprpcgen\{}
\KeywordTok{namespace} \NormalTok{Calculator\{}
\NormalTok{complex add(complex x, complex y)\{}
  \NormalTok{complex cd;}
  \NormalTok{std::complex<}\DataTypeTok{double}\NormalTok{> std_cd = std::complex<}\DataTypeTok{double}\NormalTok{>(x.r, x.i) + std::complex<}\DataTypeTok{double}\NormalTok{>(y.r, y.i);}
  \NormalTok{cd.r = std_cd.real();}
  \NormalTok{cd.i = std_cd.imag();}
  \KeywordTok{return} \NormalTok{cd;}
\NormalTok{\}}

\NormalTok{complex subtract(complex x, complex y)\{}
  \NormalTok{complex cd;}
  \NormalTok{std::complex<}\DataTypeTok{double}\NormalTok{> std_cd = std::complex<}\DataTypeTok{double}\NormalTok{>(x.r, x.i) - std::complex<}\DataTypeTok{double}\NormalTok{>(y.r, y.i);}
  \NormalTok{cd.r = std_cd.real();}
  \NormalTok{cd.i = std_cd.imag();}
  \KeywordTok{return} \NormalTok{cd;}
\NormalTok{\}}

\NormalTok{complex multiply(complex x, complex y)\{}
  \NormalTok{complex cd;}
  \NormalTok{std::complex<}\DataTypeTok{double}\NormalTok{> std_cd = std::complex<}\DataTypeTok{double}\NormalTok{>(x.r, x.i) * std::complex<}\DataTypeTok{double}\NormalTok{>(y.r, y.i);}
  \NormalTok{cd.r = std_cd.real();}
  \NormalTok{cd.i = std_cd.imag();}
  \KeywordTok{return} \NormalTok{cd;}
\NormalTok{\}}

\NormalTok{complex sum_all(std::vector<complex> contents)\{}
  \NormalTok{std::complex<}\DataTypeTok{double}\NormalTok{> currentSum(}\DecValTok{0}\NormalTok{,}\DecValTok{0}\NormalTok{);}
  \NormalTok{complex sum;}
  \KeywordTok{for}\NormalTok{(std::vector<complex>::iterator it = contents.begin(); it != contents.end(); ++it) \{}
    \NormalTok{complex current_cd = *it;}
      \NormalTok{currentSum += std::complex<}\DataTypeTok{double}\NormalTok{>(current_cd.r, current_cd.i);}
  \NormalTok{\}}
  \NormalTok{sum.r = currentSum.real();}
  \NormalTok{sum.i = currentSum.imag();}
  \KeywordTok{return} \NormalTok{sum;}
\NormalTok{\}}


\NormalTok{complex multiply_all(std::vector<complex> contents)\{}
  \NormalTok{std::complex<}\DataTypeTok{double}\NormalTok{> currentSum(}\DecValTok{1}\NormalTok{,}\DecValTok{0}\NormalTok{);}
  \NormalTok{complex sum;}
  \KeywordTok{for}\NormalTok{(std::vector<complex>::iterator it = contents.begin(); it != contents.end(); ++it) \{}
    \NormalTok{complex current_cd = *it;}
      \NormalTok{currentSum *= std::complex<}\DataTypeTok{double}\NormalTok{>(current_cd.r, current_cd.i);}
  \NormalTok{\}}
  \NormalTok{sum.r = currentSum.real();}
  \NormalTok{sum.i = currentSum.imag();}
  \KeywordTok{return} \NormalTok{sum;}
\NormalTok{\}}


\NormalTok{std::vector<}\DataTypeTok{double}\NormalTok{> map_abs(std::vector<complex> contents)\{}
  \NormalTok{std::vector<}\DataTypeTok{double}\NormalTok{> r;}
  \KeywordTok{for}\NormalTok{(std::vector<complex>::iterator it = contents.begin(); it != contents.end(); ++it) \{}
    \NormalTok{complex current_cd = *it;}
      \NormalTok{r.push_back(abs(std::complex<}\DataTypeTok{double}\NormalTok{>(current_cd.r, current_cd.i)));}
  \NormalTok{\}}
  \KeywordTok{return} \NormalTok{r;}
\NormalTok{\}}

\NormalTok{\}}
\NormalTok{\}}
\end{Highlighting}
\end{Shaded}

Without delving too much into the implementation details, what we wrote
here was all pure C++. We didn't need to use any libraries (other than
\texttt{std}), and we simply returned the results, just like we're used
to doing.

Now, all we have to do is compile and include the library in our html
file.

\hyperdef{}{building-our-rpc-module}{\subsection{Building our RPC
Module}\label{building-our-rpc-module}}

For the most part, our generated \texttt{Makefile} will do everything
for us. Depending on how complex our RPC functions are, we might need to
tweak it a bit. But for our complex number calculator project, we can
simply call \texttt{make} and everything will work.

\begin{Shaded}
\begin{Highlighting}[]
\KeywordTok{make}
\end{Highlighting}
\end{Shaded}

\begin{Shaded}
\begin{Highlighting}[]
  \KeywordTok{CXX}  \NormalTok{pnacl/Release/ComplexRPC.o}
  \KeywordTok{CXX}  \NormalTok{pnacl/Release/Calculator.o}
  \KeywordTok{LINK} \NormalTok{pnacl/Release/Complex_unstripped.bc}
  \KeywordTok{FINALIZE} \NormalTok{pnacl/Release/Complex_unstripped.pexe}
  \KeywordTok{CREATE_NMF} \NormalTok{pnacl/Release/Complex.nmf}
\end{Highlighting}
\end{Shaded}

This build process is actually included from
\texttt{\$(NACL\_SDK\_ROOT)/tools/common.mk}, which is used to build the
NaCl SDK's examples. We use it here to make it easy to change toolchain
and configuration. The default toolchain is \texttt{pnacl} and the
default config is \texttt{Release}, but we could use any of the
available toolchains (\texttt{pnacl}, \texttt{newlib}, and
\texttt{glibc}). For example, we can build the same application using
newlib by running \texttt{make TOOLCHAIN=newlib}. You can read more
about the NaCl supported toolchains
\href{https://developer.chrome.com/native-client/devguide/devcycle/building}{here}.

In the end, a \texttt{.pexe} file is generated along with the
\href{https://developer.chrome.com/native-client/reference/nacl-manifest-format}{NaCl
Manifest} (\texttt{Complex.nmf}).

Interestingly, we can package the whole Complex folder into a zip or tar
file and distribute it for any JavaScript developer to use on their
website, without even needing to compile it.

\hyperdef{}{using-our-library-from-javascript}{\subsection{Using our
library from JavaScript}\label{using-our-library-from-javascript}}

We now have a binary native client application that we can include into
our web application. To include it, we will use the Native Calls
JavaScript library. The Native Calls JavaScript library was generated
when we installed the Native Calls library using \texttt{make install}.
The generated file can be found in
\texttt{\textasciitilde{}/native-calls/scripts/build/NativeCalls.js}. We
need to put this file into our html file, along with the generated RPC
module (\texttt{complexCalculator/Complex/ComplexRPC.js}).

We copy the built \texttt{NativeCalls.js} file into our
\texttt{complexCalculator} folder, and then we can write our html file
that uses the library.

\begin{Shaded}
\begin{Highlighting}[]
\KeywordTok{cp} \NormalTok{~/native-calls/scripts/build/NativeCalls.js ./}
\KeywordTok{vim} \NormalTok{index.html}
\end{Highlighting}
\end{Shaded}

\begin{Shaded}
\begin{Highlighting}[]
\DataTypeTok{<!DOCTYPE }\NormalTok{html}\DataTypeTok{>}
\KeywordTok{<html>}
\KeywordTok{<head}\OtherTok{ lang=}\StringTok{"en"}\KeywordTok{>}
    \KeywordTok{<meta}\OtherTok{ charset=}\StringTok{"UTF-8"}\KeywordTok{>}
    \KeywordTok{<title>}\NormalTok{Complex Number Calculator}\KeywordTok{</title>}
    \KeywordTok{<script}\OtherTok{ type=}\StringTok{"text/javascript"}\OtherTok{ src=}\StringTok{"//cdnjs.cloudflare.com/ajax/libs/require.js/2.1.11/require.min.js"}\KeywordTok{></script>}
    \KeywordTok{<script>}
\ErrorTok{    require(['./Complex/ComplexRPC.js'], function (ComplexRPC) \{}
\ErrorTok{        window.Complex = new ComplexRPC();}
    \NormalTok{\});}
    \KeywordTok{</script>}
\KeywordTok{</head>}
\KeywordTok{<body>}
\KeywordTok{<h1>}\NormalTok{Complex Number Calculator}\KeywordTok{</h1>}
\KeywordTok{</body>}
\KeywordTok{</html>}
\end{Highlighting}
\end{Shaded}

Now, all that's left is to see the library in action! To do this, Native
Client requires that the files are hosted on a server. Let's install a
configure-free server such as \texttt{serve}.

\begin{Shaded}
\begin{Highlighting}[]
\KeywordTok{npm} \NormalTok{install -g serve}
\end{Highlighting}
\end{Shaded}

Great, now we can host a server in our \texttt{complexCalculator}
folder, by simply running serve.

\begin{Shaded}
\begin{Highlighting}[]
\KeywordTok{serve}
\end{Highlighting}
\end{Shaded}

\begin{Shaded}
\begin{Highlighting}[]
\KeywordTok{serving} \NormalTok{~/complexCalculator on port 3000}
\end{Highlighting}
\end{Shaded}

Now, open chrome and navigate to \texttt{http://localhost:3000/}, then
open the console to start using the library.

\hyperdef{}{making-remote-procedure-calls-from-javascript}{\subsection{Making
remote procedure calls from
JavaScript}\label{making-remote-procedure-calls-from-javascript}}

 With the console open, we can try out some remote procedure calls.
You'll notice that the functions we exposed using the idl file are
available to us in the console window. The functions work as you would
expect, but they are always completely \textbf{asynchronous} since
\texttt{postMessage} is used as the underlying transfer layer. We can
see what data is sent and received in the console. Let's call
\texttt{add} to add two numbers.

\begin{verbatim}
Complex.Calculator.add({r:10, i:10}, {r:5, i:-10}, function(result){
  console.debug(result);
});
\end{verbatim}

We can see the data being transferred in the console:

\begin{verbatim}
[NaClModule:Complex] → {"jsonrpc":"2.0","method":"Calculator::add","id":3,"params":[{"r":10,"i":10},{"r":5,"i":-10}]}

[NaClModule:Complex] ← {"id":3,"jsonrpc":"2.0","result":{"i":0,"r":15}}
\end{verbatim}

And finally the result being logged:

\begin{verbatim}
Object {i: 0, r: 15}
\end{verbatim}

This is the expected result. In fact, all remote procedure calls from
JavaScript take in an extra, optional, 2 paramaters: a success callback
and an error callback. But what happens if we do not provide the correct
number of paramaters? The RPC library is able to detect this, by
throwing an error:

\begin{verbatim}
Complex.Calculator.add();
\end{verbatim}

\begin{verbatim}
TypeError: The function add expects 2 parameter(s) but received 0
\end{verbatim}

The RPC library also has reasonable type checking. For example if we did
not pass in an object:

\begin{verbatim}
Complex.Calculator.add("hello", "world");
\end{verbatim}

\begin{verbatim}
TypeError: Parameter 0 has invalid type: string (expected object)
\end{verbatim}

Client side type-checking is also recursive:

\begin{verbatim}
Complex.Calculator.add({r:12, i:23}, {r:3 ,i:"not a number"});
\end{verbatim}

\begin{verbatim}
TypeError: Parameter 1 has invalid type: string (expected number) [at .i]
\end{verbatim}

Type checking also happens on the C++ side. In that case, the error
callback is called.

\subsubsection{Configuration}\label{configuration}

Before the module constructs, we can specify some configuration. We can
specify the toolchain, config, and whether or not we want client-side
type checking (as shown above). To do this, we edit the \texttt{script}
tag that loads the module, and set the global NaClConfig object.

\begin{Shaded}
\begin{Highlighting}[]
\KeywordTok{<script>}
\OtherTok{window}\NormalTok{.}\FunctionTok{NaClConfig} \NormalTok{= \{}
  \DataTypeTok{VALIDATION}\NormalTok{: }\KeywordTok{false} \CommentTok{// can also set TOOLCHAIN and CONFIG}
\NormalTok{\};}
\ErrorTok{require(['./Complex/ComplexRPC.js'], function (ComplexRPC) \{}
\ErrorTok{    window.Complex = new ComplexRPC();}
\NormalTok{\});}
\KeywordTok{</script>}
\end{Highlighting}
\end{Shaded}

Now, when we refresh and make a remote procedure call with incorrect
types, the callback will be called - i.e.~type checking in the C++ has
rejected the call.

\begin{verbatim}
var success = function(result){ console.log(result); };
var error = function(error){ console.error("ERROR! "+JSON.stringify(error)); };
Complex.Calculator.add({r:12,i:23},{r:3,i:"not a number"}, success, error);
\end{verbatim}

\begin{verbatim}
ERROR! {"code":-32602,"message":"Invalid Params: Param 1 (y) has incorrect type. Expected complex"}
\end{verbatim}

Turning off JS Validation can increase performance, especially for
applications that perform many requests per second.

