\section{Usability Evaluation} % (fold)
\label{sec:usability_evaluation}
To get some insight as to how usable and useful the library and generator is, we analyse the number of lines the developer would need to write to build the same application. We take a look at two different applications: a bullet physics simulation which uses the C++ module to calculate simulation steps, and a regular expression library which uses a native module to do execute regular expressions. The table below shows how many lines the developer had to write to achieve the same program.


\subsection{Bullet} % (fold)
\label{sub:evaluation_usability_bullet}
\begin{table}[h]
\begin{tabular}{llll}
\#lines                  & Original & Library & Difference \\ \cline{2-4} 
\multicolumn{1}{l|}{C++} &          &         &            \\
\multicolumn{1}{l|}{JS}  &          &         &            \\
\multicolumn{1}{l|}{IDL} & 0        &         &           
\end{tabular}
\end{table}

TODO: Describe what the code differences are and why they appeared. Describe overall process, and a conclusion.
% subsection evaluation_usability_bullet (end)

\subsection{Oniguruma} % (fold)
\label{sub:evaluation_usability_oniguruma}
\begin{table}[h]
\begin{tabular}{llll}
\#lines                  & Original & Library & Difference \\ \cline{2-4} 
\multicolumn{1}{l|}{C++} &          &         &            \\
\multicolumn{1}{l|}{JS}  &          &         &            \\
\multicolumn{1}{l|}{IDL} & 0        &         &           
\end{tabular}
\end{table}

TODO: Describe what the code differences are and why they appeared. Describe overall process, and a conclusion.
% subsection evaluation_usability_oniguruma (end)

% section usability_evaluation (end)