\chapter{Related Work} 
\label{Chapter3} 
\lhead{Chapter 3. \emph{Related Work}} 

\section{Native Client Acceleration Modules} % (fold)
\label{sec:naclam}

TODO: NaClAM. Mention:
\begin{itemize}
	\item Idea, example
	\item Design, implementation
	\item Advantages, disadvantages
\end{itemize}

% section naclam (end)

\section{Node.js C++ Bindings} % (fold)
\label{sec:node_js_c_bindings}
TODO: node.js C++ bindings. Mention:
\begin{itemize}
	\item Idea, example
	\item Typical implementation example
	\item Advantages, disadvantages
\end{itemize}
% section node_js_c_bindings (end)



\section{Apache Thrift: Cross-language services} % (fold)
\label{sec:apache_thrift_cross_language_services}
Apache Thrift is a framework that allows cross-language services development. Originally developed at Facebook, it was designed to provide reliable, efficient communication between languages and services. Many languages are supported, including C++, Java, and JavaScript. Thrift provides a cross-platform generator that can generate Thrift client and server pairs, where the client and server can be using different languages. Similar to other RPC frameworks, it uses its own IDL file format, Thrift IDL. The IDL file is used to generate code to support different languages.

An unofficial port for Apache Thrift has been made for Native Client \footnote{\url{https://github.com/ahilss/thrift-nacl}}, however, all of the communication code is still hand coded. The performance of using Thrift for Native Client is unclear, as there is no protocol implemented using PPAPI.

TODO: Better structure this section. Mention:
\begin{itemize}
	\item Idea, example.
	\item Typical implementation example, use cases.
	\item Advantages, disadvantages
\end{itemize}

% section apache_thrift_cross_language_services (end)


\section{JSON-RPC Implementations} % (fold)

TODO: Better structure this section. Include:
\begin{itemize}
	\item What they are, general design of these frameworks.
	\item Example: pmrpc (+advantages, disadvantages)
	\item General: Advantages, disadvantages
\end{itemize}

\label{sec:json_rpc_implementations}
Many JSON-RPC implementations for several languages exist\footnote{\url{http://en.wikipedia.org/wiki/JSON-RPC}}, including C++\footnote{\url{http://jsonrpc-cpp.sourceforge.net/}} and JavaScript\footnote{\url{https://github.com/gimmi/jsonrpcjs}}. However, none have been implemented for Native Client and using the PPAPI API.

\subsection{pmrpc: Inter-window JavaScript RPC using postMessage} % (fold)
\label{sub:pmrpc_json_rpc_using_postmessage}
pmrpc is an open source library available on GitHub \footnote{\url{https://github.com/izuzak/pmrpc}} which aims to simplify cross-window communication by using postMessage. It shows the simplicity of remote procedure calls using JSON-RPC but only supports browser-based JavaScript. Native Client C++ or any other language or transport is not supported.
% subsection pmrpc_json_rpc_using_postmessage (end)

% section json_rpc_implementations (end)