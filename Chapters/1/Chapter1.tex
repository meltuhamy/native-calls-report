\chapter{Introduction} 

\label{Chapter1}

\lhead{Chapter 1. \emph{Introduction}}
Over the past decades, the web has quickly evolved from being a simple online catalogue of information to becoming a massive distributed platform for web applications that are used by millions of people. Developers have used JavaScript to write web applications that run on the browser, but JavaScript has some limitations. 

One of the problems of JavaScript is performance. JavaScript is a single threaded language with lack of support for concurrency. Although web browser vendors such as Google and Mozilla are continuously improving JavaScript run time performance, it is still a slow interpreted language, especially compared to compiled languages such as C++. Many attempts have been made to increase performance of web applications. One of the first solutions was browser plugins that run in the browser, such as Flash or Java Applets. However, these have often created browser bugs and loop-holes that can be used maliciously to compromise security.

Native Client \cite{nacl} (NaCl) is a technology from Google that allows running binary code in a sandboxed environment in the Chrome browser. This technology allows web developers to write and use computation-heavy programs that run inside a web application, whilst maintaining the security levels we expect when visiting web applications.

The native code is typically written in C++, though other languages can be supported. The code is compiled and the binary application is sandboxed by verifying the code to ensure no potentially un-secure instructions or system-calls are made. This is done by compiling the source code using the gcc\footnote{The GNU Compiler Collection (gcc) is an open-source compiler that supports C, C++, and other languages \cite{gcc}} based NaCl compiler. This generates a NaCl module that can be embedded into the web page. Because no system calls can be made, the only way an application can communicate with the operating system (for example, to play audio) is through the web browser, which supports several APIs in JavaScript that are secure to use and also cross-platform. This means that the fast-performing C++ application needs to communicate with the JavaScript web application.

\lstset{language=c,caption={JavaScript code},label=javascriptcode}
\begin{lstlisting}
// Send a message to the NaCl module
function sendHello () {
  if (HelloTutorialModule) {
    // Module has loaded, send it a message using postMessage
    HelloTutorialModule.postMessage('hello');
  } else {
    // Module still not loaded!
    console.error("The module still hasn't loaded");
  }
}

// Handle a message from the NaCl module
function handleMessage(message_event) {
  console.log("NACL: " + message_event.data);
}
\end{lstlisting}


\lstset{language=c,caption={C++ Code},label=cppcode}
\begin{lstlisting}
// Handle a message coming from JavaScript
virtual void HandleMessage(const pp::Var& var_message) {
  // Send a message to JavaScript
  PostMessage(var_message);
}
\end{lstlisting}

The way Native Client modules can communicate with the JavaScript web application (and vice versa) is through simple message passing. The JavaScript web application sends a message in the form of a JavaScript string to the NaCl module. The NaCl module handles message events by receiving this string as a parameter passed into the \lstinline+HandleMessage+ function. For example, Listing \ref{javascriptcode} shows a simplified example of how JavaScript sends a message to the NaCl module, and Listing \ref{cppcode} shows how the native module handles the message and sends the same message back to the JavaScript application. This allows for straight forward, asynchronous communication between the native code and the web application. Modern web browsers support message passing using the \lstinline+postMessage+ API. This was designed to allow web applications to communicate with one or more web workers
\footnote{Web workers\cite{webworkersw3c} are scripts that run in the background of a web page, independent of the web page itself. It is a way of carrying out computations while not blocking the main page's execution. Although they allow concurrency, they are relatively heavyweight and are not intended to be spawned in large numbers. Typically a web application would have one web worker to carry out computations, and the main page to do most of the view logic (such as click listening, etc.)}. 

However, message passing puts more burden on the developer to write the required communication code between the NaCl module and the application. For example, consider a C++ program that performs some heavy computations and has functions that take several parameters of different types. To make the functionality accessible from the web application, the developer would need to write a lot of code in the \lstinline+HandleMessage+ function. A message format would need to be specified to distinguish which function is being called. Then the parameters of the function call would need to be identified, extracted, and converted into C++ types in order that the parameters are passed into the C++ function. Then a similar procedure would need to be done if the function would return anything back to the web application. 

The purpose of this project is to allow developers to easily invoke NaCl modules by creating a Remote Procedure Call (RPC) framework on top of the existing message passing mechanism. To achieve this, the developer will simply write an Interface Definition Language (IDL) file which specifies the functions that are to be made accessible from JavaScript. The IDL file will be parsed in order to automatically generate JavaScript and C++ method stubs that implement the required communication code using message passing. This is similar to how RPC is implemented in other traditional frameworks, such as ONC RPC (page \pageref{sub:oncrpc_intro}) or CORBA (page \pageref{sub:corba_intro}).

The main contributions of this project is to create a tool that parses IDL files and generates JavaScript and C++ method stubs, a message format that will be used in communication, and support libraries in JavaScript and C++ that will use message passing to do the actual communication. This will allow functions in the Native Client module to be called directly from the JavaScript application. We will evaluate how much this will help developers by seeing how many lines can be saved, in different program contexts. We will also analyse the speed and efficiency of using RPC over hand-written message passing. 
