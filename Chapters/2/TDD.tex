\section{Test Driven Development} % (fold)
\label{sec:test_driven_development}
Test Driven Development (TDD) is a software development approach whereby the developer writes unit-tests that describe some functionality first, then implements the functionality in order to make the tests pass.

The project includes several tests for each component of the system. Unit tests are written to test fine-grained functionality (e.g. functions), while end-to-end (E2E) tests have been written to test large parts of the system as a whole.

Because the project is implemented on both C++ and JavaScript, tests had to be written for each of these languages. Thus, the project includes the following tests:

\begin{itemize}
	\item JavaScript library tests: These test the functionality of each component of the JavaScript library. The tests run on the browser.
	\item Generator tests: These test the functionality of the JS and C++ generators.
	\item C++ library tests: These test the functionality of the C++ RPC library.
	\item E2E tests: These are test applications written to test the `full stack': starting from code generation, compilation, and all the way down to individual RPC call requests.
\end{itemize}

\subsection{Karma Test Runner} % (fold)
\label{sub:karma_test_runner}
Karma test runner\footnote{\url{http://karma-runner.github.io/}} is a library implemented at Google that makes running JavaScript tests easy. It was designed to simplify and speed up test-driven development for JavaScript. It works by letting the developer specify a configuration that states which files should be loaded, then the tests are run directly from the command line. This means the developer does not need to open a browser manually every time they want to run the tests.

Karma has been used extensively in this project to test the client side JavaScript and C++. More about configuring Karma to load C++ tests is described in the design section.
%TODO: link

% subsection karma_test_runner (end)

\subsection{JavaScript Testing Framework} % (fold)
\label{sub:js_test_framework}
Many testing frameworks exists for JavaScript. The most popular are Jasmine\footnote{\url{http://jasmine.github.io/}}, QUnit\footnote{\url{http://qunitjs.com/}} and Mocha\footnote{\url{http://visionmedia.github.io/mocha/}}. All of them support a similar set of features and APIs. 

\begin{itemize}
	\item Jasmine: %TODO
	\item QUnit:  %TODO
	\item Mocha: %TODO
\end{itemize}

In the end, I decided to use Jasmine for its straight-forward set up and easy configuration with Karma.

% subsection js_test_framework (end)

\subsection{C++ Testing Framework} % (fold)
\label{sub:cpp_testing}
Again, many unit testing frameworks and libraries exist for C++. The most popular are CppUnit, googletest, and the Boost test library.

\begin{itemize}
	\item CppUnit
	\item googletest
	\item Boost
\end{itemize}

Google Mock (gmock) is a powerful library that allows mocking classes in C++. Mocking makes it easier to test different components of a system without requiring the actual implementation. gmock can be used with any testing framework, and it is one of the most mature mocking libraries for C++ available.

I decided to use googletest for its simpler syntax, portability (as it was included with the NaCl SDK) and the fact it was easy to integrate gmock for mocking classes.

% subsection c_testing_using_gtest_and_gmock (end)

% section cpp_testing (end)