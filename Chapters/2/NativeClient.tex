\section{Native Client}
Native Client (NaCl) can be thought of as a new type of plugin for the Google Chrome browser that allows binary programs to run natively in the web browser. It can be used as a `back end' for a normal web application written in JavaScript, since the binary program will run much faster. A NaCl module can be written in any language, including Assembly, so long as the binary is checked and verified to be safe by the NaCl sandbox \cite{nacl}. However, NaCl provides a Software Development Kit (SDK) that includes a compiler based on gcc that allows developers to compile C and C++ programs into binary that will work directly with the sandbox without further modifications. Thus, writing NaCl-compatible C++ programs is just as easy as writing normal C++ programs. The difference is that the sandboxes disallow unwanted side-effects, including system-calls. Since many applications might want to have these side effects, Native Client provides a set of cross-platform API functions that achieve the same outcomes, but by communicating with the browser directly. To avoid calling NaCl syscalls directly, an independant runtime (IRT) is provided, along with two different C libraries (newlib and glibc) on top of which the Pepper Plugin API (PPAPI or `Pepper') is exposed. It can be used to do file IO, play audio, and render graphics. The PPAPI also includes the \lstinline+PostMessage+ functionality, which allows the NaCl module to communicate with the JavaScript application.

\subsection{NaCl Modules and the Pepper API}
A native client application consists of the following \cite{nacloverview}:
\begin{description}
  \item[HTML/JavaScript Application] 
  This is where the user interface of the application will be defined, and the   JavaScript here could also perform computations. The HTML file will include   the NaCl module by using an embed tag, e.g. \\
   \lstinline+<embed src="myModule.nmf" type="application/x-nacl" />+
  \item[Pepper API] 
  Allows the NaCl module communicate with the web browser and use its features. Provides \lstinline+PostMessage+ to allow message passing to the JavaScript application.
  \item[Native Client Module] 
  The binary application, which performs heavy computation at native speeds.
\end{description}

\subsection{Communicating with JavaScript using postMessage}
The HTML5 \lstinline+postMessage+ API was designed to allow web workers to communicate with the main page's JavaScript execution thread. The JavaScript object is copied to the web worker by value. If the object has cycles, they are maintained as long as the cycles exist in the same object. This is known as the structured clone algorithm, and is part of the HTML5 draft specification \cite{html5w3c}. 

In a similar way, \lstinline+postMessage+ allows message passing to and from NaCl modules. However, sending objects with cycles will cause an error. NaCl allows sending and receiving primitive JavaScript objects (\lstinline+Numbers+, \lstinline+String+s, \lstinline+Boolean+s, \lstinline+null+) as well as dictionaries (key-value \lstinline+Object+s), arrays, and \lstinline+ArrayBuffers+. ArrayBuffers are a new type of JavaScript object based on Typed Arrays \cite{typedarraysw3c} that allows the storing of binary data. 

Another key difference is that message types need to be converted into the correct type on the receiving end. For example, sending a JavaScript \lstinline+Object+ should translate into a dictionary type. The JavaScript types are dynamic in nature. A JavaScript \lstinline+Number+ object could be an integer, a float, a double, `infinity', exponential, and so on. Sending C++ data to JavaScript is simple since it is converting from a more specific type to a less specific type (e.g. from \lstinline+int+ in C++ to \lstinline+Number+ in JavaScript). But converting from a JavaScript type to a C++ type requires more thought. The PPAPI provides several functions to determine the JavaScript type (e.g. \lstinline+bool is_double()+). It also allows us to extract and cast the data into our required type (e.g. \lstinline+double AsDouble()+). From there, we can use the standard C++ type to perform the required computations.
